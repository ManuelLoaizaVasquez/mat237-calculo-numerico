\begin{statement}{2}
  Dada la funci\'on $f: \BR \to \BR$ con $f(x) = e^{-x}$, deseamos calcular num\'ericamente
  \[
    \int_0^{0.8} f(x) dx.
  \]
  Usaremos los valores de $f(x)$ en los puntos $0$, $0.2$, $0.4$, $0.6$ y $0.8$.
  Genere la data antes de comenzar la integraci\'on num\'erica.
\end{statement}

\begin{statement}{a}
  Escriba la regla trapezoidal y calcule la integraci\'on num\'erica con seis d\'igitos.
\end{statement}

\begin{solution}
  Aproximamos la integral v\'ia
  \[
    I(f) \approx T(f; h) =
    h \left(\frac{1}{2}(f(x_0) + f(x_4)) + \sum_{i = 1}^3 f(x_i)\right),
  \]
  donde $n = 4$, $h = 0.2$ y $0 = x_0 < 0.2 = x_1 < 0.4 = x_2 < 0.6 = x_3 < 0.8 = x_4$.
  Implementamos el algoritmo
  \lstinputlisting{scripts/algorithms/trapezoid-rule.m}
  y lo llamamos
  \lstinputlisting{scripts/problems/problem-02-01.m}
  obteniendo $0.552505$.
\end{solution}

\begin{statement}{b}
  Escriba la regla de Simpson y calcule la integraci\'on num\'erica con seis d\'igitos.
\end{statement}

\begin{solution}
  Aproximamos la integral v\'ia
  \[
    I(f) \approx S(f; h) = \frac{h}{3} \left(f(x_0) + f(x_8) + 4 \sum_{i = 1}^4 f(x_{2i - 1}) + 2 \sum_{i = 1}^3 f(x_{2i})\right),
  \]
  donde $n = 4$, $h = 0.1$ y $x_0 = 0 < x_2 < 0.2 < x_4 = 0.4 < x_6 = 0.6 < x_8 = 0.8$.
  Implementamos el algoritmo
  \lstinputlisting{scripts/algorithms/simpson-rule.m}
  y lo llamamos
  \lstinputlisting{scripts/problems/problem-02-02.m}
  obteniendo $0.550671$.
\end{solution}

\begin{statement}{c}
  ¿Cu\'al es el valor exacto de la integral?
  ¿Cu\'al es el error absoluto usando la regla trapezoidal y la de Simpson?
  ¿Cu\'al m\'etodo es mejor?
\end{statement}

\begin{solution}
  Integramos
  \[
      \int_0^{0.8} e^{-x} \, dx = -e^{-x} \Big|_0^{0.8} = 1 - e^{-0.8} \approx 0.550671.
  \]
  El error absoluto, con seis d\'igitos,
  usando la regla trapezoidal es $E_T(f; h) = 0.001834$
  mientras que usando la regla de Simpson es $E_S(f, h) = 0$.
  Debido a que hemos obtenido un menor error absoluto,
  podemos decir que el m\'etodo de Simpson
  fue mejor para esta integral num\'erica.
\end{solution}

\begin{statement}{d}
  La f\'ormula del error para la regla trapezoidal con $n + 1$ puntos es
  \[
    E_T(f; h) = -\frac{b - a}{12} h^2 f''(\xi),\; h = \frac{b - a}{n},
  \]
  para alg\'un $\xi \in [a, b]$.

  El error para la regla de Simpson con $2n + 1$ puntos es
  \[
    E_S(f; h) = -\frac{b - a}{180} h^4 f^{(4)}(\xi),\; h = \frac{b - a}{2n},
  \]
  para alg\'un $\xi \in [a, b]$.

  Si deseamos que el valor absoluto del error sea menor que $10^{-4}$,
  ¿cu\'antos puntos se necesitar\'an para cada m\'etodo?
\end{statement}

\begin{solution}
  Como la funci\'on $e^{-x}$ es decreciente en todo su dominio, el m\'aximo
  valor que puede tomar en $[0, 0.8]$ es $1$.

  Determinemos la m\'inima cantidad de puntos necesarios para que el error absoluto
  tras usar la regla trapezoidal sea menor que $10^{-4}$.
  Tenemos $f''(x) = e^{-x}$. Luego
  \[
    |E_T(f; h)| \leq \frac{0.8 - 0}{12} \cdot 1 \cdot h^2 < 10^{-4}.
  \]
  Manipulemos esta expresi\'on
  \begin{align*}
    h^2 &< \frac{12}{0.8} \cdot 10^4 \\
    &= 0.0015 \\
    h &< 0.0387298 \\
    \frac{0.8 - 0}{n} &< 0.0387298 \\
    n &> 20.6559.
  \end{align*}
  Escogemos $n = 21$ y por lo tanto $n + 1 = 22$ puntos.

  Ahora determinemos la m\'inima cantidad de puntos necesarios para que el error absoluto
  tras utilizar la regla de Simpson sea menor que $10^{-4}$.
  Tenemos $f^{(4)}(x) = e^{-x}$. Luego
  \[
    |E_S(f; h)| \leq \frac{0.8 - 0}{180} \cdot 1 \cdot h^4 < 10^{-4}.
  \]
  Manipulemos esta expresi\'on
  \begin{align*}
    h^4 &< \frac{180}{0.8 \cdot 10^4} \\
    h &< 0.3872983 \\
    \frac{0.8 - 0}{2n} &< 0.3872983 \\
    n &> 1.03280.
  \end{align*}
  Escogemos $n = 2$ y por lo tanto $2n + 1 = 5$ puntos.
\end{solution}

