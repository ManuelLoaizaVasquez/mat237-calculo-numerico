\begin{statement}{4}
  Considere la regla de cuadratura gaussiana con cuatro puntos en el intervalo $[-1, 1]$
  \[
    \int_{-1}^1 f(x) \, dx \approx a_1 f(x_1) + a_2 f(x_2) + a_3 f(x_3) + a_4 f(x_4)
  \]
  donde
  \begin{align*}
    x_1 &= -\sqrt{(3 - 4 \sqrt{0.3}) / 7}, \\
    x_2 &= -\sqrt{(3 + 4 \sqrt{0.3}) / 7}, \\
    x_3 &= \sqrt{(3 - 4 \sqrt{0.3}) / 7}, \\
    x_4 &= \sqrt{(3 + 4 \sqrt{0.3}) / 7}, \\
    a_1 = a_3 &= 1/2 + \sqrt{10 / 3} / 12, \\
    a_2 = a_4 &= 1/2 - \sqrt{10 / 3} / 12.
  \end{align*}
  Pruebe que la regla es exacta para todos los polinomios de grado menor o igual a siete.
\end{statement}

\begin{solution}
  Verifiquemos que la regla satisface la igualdad para $f(x) = 1, x, x^2$ y $x^3$.
  El caso $f(x) = 1$ es f\'acil de verificar
  \begin{align*}
    2 &= \left(\frac{1}{2} + \frac{1}{12}\sqrt{\frac{10}{3}}\right)
    + \left(\frac{1}{2} - \frac{1}{12}\sqrt{\frac{10}{3}}\right)
    + \left(\frac{1}{2} + \frac{1}{12}\sqrt{\frac{10}{3}}\right)
    + \left(\frac{1}{2} - \frac{1}{12}\sqrt{\frac{10}{3}}\right).
  \end{align*}
  Para $f(x) = x$ trivialmente la suma se anula
  \begin{align*}
    0 &= \left(\frac{1}{2} + \frac{1}{12}\sqrt{\frac{10}{3}}\right) \cdot \left(- \sqrt{\frac{1}{7}\left(3 - 4 \sqrt{\frac{3}{10}}\right)}\right)
    + \left(\frac{1}{2} - \frac{1}{12}\sqrt{\frac{10}{3}}\right) \cdot \left(- \sqrt{\frac{1}{7}\left(3 + 4 \sqrt{\frac{3}{10}}\right)}\right)\\
    &\left(\frac{1}{2} + \frac{1}{12}\sqrt{\frac{10}{3}}\right) \cdot \left(\sqrt{\frac{1}{7}\left(3 - 4 \sqrt{\frac{3}{10}}\right)}\right)
    + \left(\frac{1}{2} - \frac{1}{12}\sqrt{\frac{10}{3}}\right) \cdot \left(\sqrt{\frac{1}{7}\left(3 + 4 \sqrt{\frac{3}{10}}\right)}\right).
  \end{align*}
  Para $f(x) = x^2$ obtenemos la igualdad tras operar
  \begin{align*}
    \frac{2}{3} &= \left(\frac{1}{2} + \frac{1}{12}\sqrt{\frac{10}{3}}\right) \cdot \left(\frac{1}{7}\left(3 - 4 \sqrt{\frac{3}{10}}\right)\right)
    + \left(\frac{1}{2} - \frac{1}{12}\sqrt{\frac{10}{3}}\right) \cdot \left(\frac{1}{7}\left(3 + 4 \sqrt{\frac{3}{10}}\right)\right)\\
    &\left(\frac{1}{2} + \frac{1}{12}\sqrt{\frac{10}{3}}\right) \cdot \left(\frac{1}{7}\left(3 - 4 \sqrt{\frac{3}{10}}\right)\right)
    + \left(\frac{1}{2} - \frac{1}{12}\sqrt{\frac{10}{3}}\right) \cdot \left(\frac{1}{7}\left(3 + 4 \sqrt{\frac{3}{10}}\right)\right)\\
    &= \frac{2}{7}
    \left[
      \left(
        \frac{1}{2} + \frac{1}{12}\sqrt{\frac{10}{3}}
      \right)
      \left(
        3 - 4 \sqrt{\frac{3}{10}}
      \right)
      +
      \left(
        \frac{1}{2} - \frac{1}{12}\sqrt{\frac{10}{3}}
      \right)
      \left(
        3 + 4 \sqrt{\frac{3}{10}}
      \right)
    \right] \\
    &= \frac{2}{7}
    \left(
      \frac{3}{2} - 2 \sqrt{\frac{3}{10}} + \frac{1}{4} \sqrt{\frac{10}{3}} - \frac{1}{3} + \frac{3}{2} + 2 \sqrt{\frac{3}{10}} - \frac{1}{4} \sqrt{\frac{10}{3}} - \frac{1}{3}
    \right) \\
    &= \frac{2}{7} \left(3 - \frac{2}{3}\right)\\
    &= \frac{2}{3}.
  \end{align*}
  Para $f(x) = x^3$, al igual que en el caso lineal, debido a la simetr\'ia y al ser esta una funci\'on impar, trivialmente se anula.

  Finalmente, podemos concluir que la regla es exacta para todos los polinomios de grado menor
  o igual a $2n + 1 = 2 \cdot 3 + 1 = 7$.
\end{solution}
